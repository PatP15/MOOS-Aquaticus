\documentclass[11pt]{article}
\author{Raphael Segal}

\usepackage{amsmath}
\usepackage{amssymb}
\usepackage{fancyhdr}

\usepackage{graphicx}

\usepackage{amsfonts}
\usepackage{amsthm}
\usepackage{enumerate}% http://ctan.org/pkg/enumerate
\usepackage{hyperref}
\usepackage{fixltx2e}
\usepackage{csquotes}
\usepackage{cancel}
\usepackage{listings}

\oddsidemargin0cm
\topmargin-2cm     %I recommend adding these three lines to increase the 
\textwidth16.5cm   %amount of usable space on the page (and save trees)
\textheight23.5cm  

\newcommand{\question}[2] {\vspace{.25in} \hrule\vspace{0.5em}
\noindent{\bf #1: #2} \vspace{0.5em}
\hrule \vspace{.10in}}
\renewcommand{\part}[1] {\vspace{.10in} {\bf (#1)}}

\newcommand{\OPT}{\textit{OPT}}
\newcommand{\OPTlp}{\textit{OPT\texorpdfstring{\textsubscript{LP}}}}
\newcommand{\cmdline}[1]{\vspace{.5em} $\:$\\ \begin{minipage}{\dimexpr\textwidth-2cm}
\texttt{#1}
\end{minipage} $\:$\\ \vspace{-.5em} }
\newcommand{\course}{\texttt{MIT 2.680} }
\newcommand{\lab}{PAVLAB }

\setlength{\parindent}{0pt}
\setlength{\parskip}{5pt plus 1pt}
\title{uFleetManager Guide}
 
\begin{document}
\maketitle
\section{Purpose}
This guide is intended to explain to a member of the PAVLAB how to use and improve the fleet manager app. Those members are expected to have a basic familiarity with MOOS-IvP, C++, and Bash.
\section{Usage}
\subsubsection{Dependencies}
\texttt{uFleetManager} was developed for Mac. It is in principle compatible with Linux, but that has never been demonstrated.

Currently, the only dependency is \texttt{ncurses}. On a Mac, use Macports or Homebrew like so:
\cmdline{port install ncurses}
The usage is slightly more complicated on Linux. Without having gotten it working, it's hard to say for sure, but it looks like \texttt{libncurses5-dev} is the correct version. So on Ubuntu, the usage would be
\cmdline{apt-get install libncurses5-dev}
\subsection{Installation}
\texttt{uFleetManager} is bundled in the \texttt{moos-ivp-aquaticus} tree. Assuming you haven't already, install \texttt{moos-ivp-aquaticus} in your home directory.
\subsubsection{Download ARO}
Most users will use the Anonymous Read Only version of \texttt{moos-ivp-aquaticus}: 
\cmdline{svn co https://oceanai.mit.edu/svn/moos-ivp-aquaticus-aro-trunk/trunk moos-ivp-aquaticus}
\subsubsection{Download for Editing}
A few users will have edit and commit privileges; speak to Dr. Benjamin.
\subsubsection{Enable}
Open \texttt{$\sim$/moos-ivp-aquaticus/src/CMakeLists.txt} and in the \texttt{BUILD$\_$ALL} section, find the line \texttt{ADD$\_$SUBDIRECTORY(uFleetManager)} and uncomment it. Remember to recomment it before committing code, and check it after pulling down new code.
\subsection{Running the Fleet Manager}
Build the fleet manager with the aquaticus build script:
\cmdline{$\sim$/moos-ivp-aquaticus/build.sh}
Run it with
\cmdline{$\sim$/moos-ivp-aquaticus/bin/uFleetManager}
With no arguments, the fleet manager will monitor all known machines, but not be able to launch missions. To add a mission, write a config file (see Config Files) and include it with \texttt{--file}:
\cmdline{$\sim$/moos-ivp-aquaticus/bin/uFleetManager --file /path/to/foobar.moos}
\subsection{Views}
\begin{tabular}{r|cl}
View Name & Nav Key & Description\\
\hline
Main & m & Main window, provides a ready/not ready summary of vehicle state.\\
Network & n & Vehicle addresses and whether ping and ssh test succeed\\
SVN Revisions & v & Lists revisions and summarizes which trees are most up-to-date for\\&& \texttt{moos-ivp}, \texttt{moos-ivp-aquaticus}, \texttt{moos-ivp-colregs}, and \texttt{pablo-common}\\
Command History & H & Lists the commands dispatched by the operator\\
MOOS-IvP & M & Lists mission configuration and details about the specified mission\\
\end{tabular}
\subsection{Commands}
\subsection{Config files}
\section{Modifying the Fleet Manager}
\subsection{Adding New Windows}
\subsection{Adding New Topics to a Window}
\subsection{Adding New Commands}
\subsubsection{Standard PAVLAB Commands}
\subsubsection{Special Commands}
\end{document}