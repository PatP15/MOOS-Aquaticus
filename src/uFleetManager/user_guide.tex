\documentclass[11pt]{article}
\author{Raphael Segal}

\usepackage{amsmath}
\usepackage{amssymb}
\usepackage{fancyhdr}

\usepackage{graphicx}

\usepackage{amsfonts}
\usepackage{amsthm}
\usepackage{enumerate}% http://ctan.org/pkg/enumerate
\usepackage{hyperref}
\usepackage{fixltx2e}
\usepackage{csquotes}
\usepackage{cancel}
\usepackage{listings}

\oddsidemargin0cm
\topmargin-2cm     %I recommend adding these three lines to increase the 
\textwidth16.5cm   %amount of usable space on the page (and save trees)
\textheight23.5cm  

\newcommand{\question}[2] {\vspace{.25in} \hrule\vspace{0.5em}
\noindent{\bf #1: #2} \vspace{0.5em}
\hrule \vspace{.10in}}
\renewcommand{\part}[1] {\vspace{.10in} {\bf (#1)}}

\newcommand{\OPT}{\textit{OPT}}
\newcommand{\OPTlp}{\textit{OPT\texorpdfstring{\textsubscript{LP}}}}
\newcommand{\cmdline}[1]{\vspace{.2em} $\:$\\ \begin{minipage}{\dimexpr\textwidth-2cm}
\texttt{#1}
\end{minipage} $\:$\\ \vspace{-.2em} }
\newcommand{\course}{\texttt{MIT 2.680} }
\newcommand{\lab}{PAVLAB }

\setlength{\parindent}{0pt}
\setlength{\parskip}{5pt plus 1pt}
\title{uFleetManager Guide}
 
\begin{document}
\maketitle
\section{Purpose}
This guide is intended to explain to a member of the PAVLAB how to use and improve the fleet manager app. Those members are expected to have a basic familiarity with MOOS-IvP, C++, and Bash.
\section{Usage}
\subsection{Dependencies}
\texttt{uFleetManager} was developed for Mac. It is in principle compatible with Linux, but that has never been demonstrated.

Currently, the only dependency is \texttt{ncurses}. On a Mac, get it with either Macports or Homebrew;
\cmdline{port install ncurses}
The usage is slightly more complicated on Linux. Without having gotten it working, it's hard to say for sure, but it looks like \texttt{libncurses5-dev} is the correct version. So on Ubuntu, get it with
\cmdline{apt-get install libncurses5-dev}
\subsection{Installation}
\texttt{uFleetManager} is bundled in the \texttt{moos-ivp-aquaticus} tree. Assuming you haven't already, install \texttt{moos-ivp-aquaticus} in your home directory.
\subsubsection{Download ARO}
Most users will use the Anonymous Read Only version of \texttt{moos-ivp-aquaticus}: 
\cmdline{svn co https://oceanai.mit.edu/svn/moos-ivp-aquaticus-aro-trunk/trunk moos-ivp-aquaticus}
\subsubsection{Download for Editing}
A few users will have edit and commit privileges; speak to Dr. Benjamin. Then get the codebase with
\cmdline{svn co svn+ssh://[YOUR\_USERNAME]@oceanai.mit.edu/home/svn/repos/moos-ivp-aquaticus/trunk moos-ivp-aquaticus}
\subsubsection{Enable}
At the time of writing, this code is considered experimental and therefore disabled by default.

Open \texttt{$\sim$/moos-ivp-aquaticus/src/CMakeLists.txt} and in the \texttt{BUILD$\_$ALL} section, find the line \texttt{ADD$\_$SUBDIRECTORY(uFleetManager)} and uncomment it. Remember to recomment it before committing code, and check it after pulling down new code.
\subsection{Starting the Fleet Manager}
Build the fleet manager with the normal aquaticus build script, e.g.:
\cmdline{$\sim$/moos-ivp-aquaticus/build.sh}
You'll be looking for the \texttt{uFleetManager} executable in \texttt{$\sim$/moos-ivp-aquaticus/bin} 

There are two use cases; with or without a config file. You can run \texttt{uFleetManager} with no arguments, e.g.
\cmdline{$\sim$/moos-ivp-aquaticus/bin/uFleetManager}
and that will let you observe all the lab vehicles but wont let you launch a MOOS-IvP mission.

To only observe specific vehicles, and to run a particular MOOS-IvP mission, you'll want to run it with a config file;
\cmdline{$\sim$/moos-ivp-aquaticus/bin/uFleetManager --file /path/to/my\_config\_file.moos}
Refer to the Config Files section for how to write a configuration file.

\subsection{Fleet Maanger Layout}
Once you start the Fleet Manager, you should see something like this, without the highlight colors.

\includegraphics[scale=0.5]{/Users/raphaelsegal/Desktop/uFleetManager_Layout.png}
\begin{description}
\item[Blue] Header; displays the state of the app
\item[Red] Window; displays the view indicated in the header (see below for details on each view).
\item[Yellow] Help; displays the currently available command set.
\item[Green] My Machine; displays own computer information. 'Time' is the one topic that is expected to change frequently and consistently, and therefore can be used to determine if the app has crashed.
\item[Grey] Last command; displays the last command issued. 
\item[Purple] Input; shows currently input characters
\end{description}
\subsubsection{Header}
\includegraphics[scale=0.6]{/Users/raphaelsegal/Desktop/uFleetManager_Header.png}

Displays three state variables: the current view, whether verbosity is toggled on or off, and whether commanding is toggled on or off. Sections with multiple levels of verbosity have an asterisk after their headers, and will be noted below.
\subsubsection{Windows}
See the Views section below.
\subsubsection{Help}
\includegraphics[scale=0.6]{/Users/raphaelsegal/Desktop/uFleetManager_HelpBasic.png}

The minimal set of options.

\includegraphics[scale=0.6]{/Users/raphaelsegal/Desktop/uFleetManager_HelpCommon.png}

The full set of options outside of commanding mode; contains the common set, navigation commands, and the command to clear the local cache of information requests.

\includegraphics[scale=0.6]{/Users/raphaelsegal/Desktop/uFleetManager_HelpCommands.png}

The full set of options, including those in commanding mode.

\subsubsection{My Machine}
\includegraphics[scale=0.6]{/Users/raphaelsegal/Desktop/uFleetManager_Self.png}

Stats about your own machine. \texttt{Time} serves as a responsive UI element, demonstrating that the app is actually refreshing. \texttt{MY IP} is helpful if you're running the shoreside on your computer and need to update the machine's UI, but it also indicates which wifi network you're on; if the first block is 10 you're probably on MIT-GUEST, if the wifi is 192.168.1.X, you're probably on kayak-local. You will only be able to talk to the robots on kayak-local.
\subsubsection{Footer}
\includegraphics[scale=0.6]{/Users/raphaelsegal/Desktop/uFleetManager_Footer.png}

Information about keys you're currently inputting, and the executive summary of the command you've most recently input.
\subsection{Views}
\begin{tabular}{r|cl}
View Name & Nav Key & Description\\
\hline
Main & m & Main window, provides a ready/not ready summary of vehicle state.\\
Network & n & Vehicle addresses and whether ping and ssh test succeed\\
SVN Revisions & v & Lists revisions and summarizes which trees are most up-to-date for\\&& \texttt{moos-ivp}, \texttt{moos-ivp-aquaticus}, \texttt{moos-ivp-colregs}, \texttt{pablo-common},\\
&& and \texttt{mokai-common}\\
Command History & H & Lists the commands dispatched by the operator\\
MOOS-IvP & M & Lists mission configuration and details about the specified mission\\
Previous & p & Go to previous view\\
\end{tabular}
\subsubsection{Main}
\includegraphics[scale=0.4]{/Users/raphaelsegal/Desktop/uFleetManager_Main.png}

\begin{tabular}{l|ll}
Topic & Explanation & Comments\\
\hline
M\# & Machine \#; the \# in the Commands section & Limited to $0<=M\#<10$\\
Name & Vehicle Name & List hard coded in Configuration class\\
ID & Lab vehicle id system, alpha=1, bravo=2, ... & \\
SVN & Summary; worst status from all its svn trees & OLD and NEW are relative amongst vehicles\\
&& e.g. if even one of your trees is out of date,\\
&& then your summary will be OLD. \\ && See SVN view for more detail\\
F NET & Front Seat network summary & ssh and ping; see Network view for more detail\\
B NET & Back Seat network summary & same as F NET.\\
&& Single-computer robots are back seats\\
COMPASS & Reports if vehicle's compass is up & M300 common failure mode is NaNs\\&&Mokai common failure mode is disconnects\\
GPS & Reports GPS status & M300 reports PDOP\\&& Mokai only reports connectedness\\
MOOSDB &Counts the MOOSDB processes running & 1 is the only sane value\\&&Also lists the vehicle's team, if one is given; \\&& see the MOOS-IvP section\\
\end{tabular}
\subsubsection{Network}
\includegraphics[scale=0.4]{/Users/raphaelsegal/Desktop/uFleetManager_Net.png}

\begin{tabular}{l|ll}
Topic & Explanation & Comments\\
\hline
M\# && See the Main view\\
Name && See the Main view\\
ID && See the Main view\\
F & Front Seat block & \\
PING & Is ADDR reachable by ping & NA indicates no front seat expected\\
SSH & If USER@ADDR can run a simple test command & NA indicates no front seat expected\\
USER & The front seat username\\
ADDR & The front seat address\\
B & Back Seat block & Single-computer vehicles are considered \\&& back seats\\
PING & Is ADDR reachable by ping & \\
SSH & If USER@ADDR can run a simple test command & \\
USER & The back seat username\\
ADDR & The back seat address\\
\end{tabular}
\subsubsection{SVN}
\includegraphics[scale=0.4]{/Users/raphaelsegal/Desktop/uFleetManager_Svn.png}

\begin{tabular}{l|ll}
Topic & Explanation & Comments\\
\hline
M\# && See the Main view\\
Name && See the Main view\\
ABC REV & Revision number of the copy of ABC & \\
ABC CMP & ABC tree is comparatively OLD or NEW(est) & Contacting a new machine may change \\&&who is newest\\
\end{tabular}
The tracked trees are \texttt{moos-ivp}, \texttt{moos-ivp-aquaticus}, \texttt{moos-ivp-colregs}, \texttt{pablo-common} and \texttt{mokai-common}. The PABLO and Mokai trees tend to not coexist, so they are special cased on the Main view such that having one but not the other will not bubble up an error.
\subsubsection{History}
\includegraphics[scale=0.4]{/Users/raphaelsegal/Desktop/uFleetManager_Hist.png}

\begin{tabular}{l|ll}
Topic & Explanation & Comments\\
\hline
EXEC SUMMARY & Explains the command & Most recent is is displayed in the footer\\
TIME & Time command was dispatched & Local computer time\\
Full Command & Full command as sent over the wire & Often very large; toggle verbosity to read
\end{tabular}

Only the last ten commands are displayed. 

Note: the astersik in the header indicates that the topic has multiple verbosity modes.
\subsubsection{MOOS-IvP}
\includegraphics[scale=0.4]{/Users/raphaelsegal/Desktop/uFleetManager_Moos.png}

\begin{tabular}{l|ll}
Topic & Explanation & Comments\\
\hline
M\# && See the Main view\\
Name && See the Main view\\
ID && See the Main view\\
A & Actual results block & Values here read off the target machine\\
MOOSDB && See the Main view\\&&Note that team isn't included here, unlike in the Main view\\
E & Expected results block & Values here are what uFleetManager would dispatch\\
Team & Team that the machine is on & Read from config\\
Mission & Launch file and args & If this is blank, startMOOS doesn't dispatch anything\\ && Toggle verbosity to see full path\\
\end{tabular}

Note: the astersik in the header indicates that the topic has multiple verbosity modes.

\subsection{Commands}
Many of these commands require the operator's fleet manager to be in "commanding mode"; they will be indicated by a * next to their name in this list. Some of these commands require confirmation; they will be indicated with a \$ next to their name in this list.

\begin{tabular}{r|ll}
Command & Key Feed & Description\\
\hline
Quit & ctrl-c & Close uFleetManager\\
Help & ctrl-h & Toggle help text; default is most hidden\\
CMD mode & ctrl-a & Toggle command mode; default is not in command\\
Verbose mode & V & Toggle verbose mode; default is terse\\
Clear & Backspace & Clear key feed\\ 
Start MOOS* & S & Start MOOS on each available machine, if possible\\
& s\# & Start MOOS on machine \#, if possible\\
Stop MOOS*\$ & K & Stop MOOS on all available machines (aka \texttt{ktm})\\
& k\# & Stop MOOS on machine \#\\
Restart MOOS*\$ & R & Equivalent to the sequence \texttt{K S}\\
& r\# & Equivalent to the sequence \texttt{k\# s\#}\\
Reboot Machine*\$ & W & Reboot all the machines (back seats)\\
& w\# & Reboot machine \#'s back seat\\
Shutdown Machine*\$ & D & Shutdown all the machines (back seats)\\
& d\# & Shutdown machine \#'s back seat\\
Reboot Vehicle*\$ & G & Reboot each of the machines' front seats, if they have them\\
& g\# & Reboot machine \#'s front seat, if it has one\\
Shutdown Vehicle*\$ & F & Shutdown each of the machines' front seats, if they have them\\
& f\# & Shutdown machine \#'s front seat, if it has one
\end{tabular}
\subsection{Config files}
Config files use standard \texttt{.moos} file syntax. The minimal config file looks like this

\cmdline{ProcessConfig = uFleetManager}
\cmdline{\{}
\cmdline{	machines = alpha bravo charlie}
\cmdline{\}}

and this would direct the Fleet Manager to watch three known vehicles; \texttt{Alpha}, \texttt{Bravo}, and \texttt{Charlie}, without specifying their mission or team.

To specify their team, add team variables like so:

\cmdline{ProcessConfig = uFleetManager}
\cmdline{\{}
\cmdline{	machines = alpha bravo charlie}
\cmdline{	red = alpha}
\cmdline{	blue = bravo delta}
\cmdline{\}}

This will result in \texttt{Alpha} and \texttt{Bravo} being assigned teams (which will show up in the \texttt{main} and \texttt{MOOS} views), \texttt{Charlie} will not have a team, and \texttt{Delta}'s team would be ignored because \texttt{Delta} isn't called out in \texttt{machines}.

To specify a mission directory, add the full path;

\cmdline{	all\_mission\_dir = $\sim$/some/fully/qualified/path/}

To specify vehicle-specific arguments for vehicle \texttt{foobar}, add a variable with its name;

\cmdline{	foobar = launch:launch.sh, dir:some/other/path, dir\_rel:true, args:-s}

Which would instruct \texttt{foobar} to use the \texttt{launch.sh} launch file when given the command to launch MOOS-IvP by the Fleet Manager, and to look for it at \texttt{$\sim$/some/fully/qualified/path/some/other/path}, and pass it the argument \texttt{-s}. If \texttt{dir\_rel} was given \texttt{false}, then the Fleet Manager would go instead to \texttt{some/other/path} and look for \texttt{launch.sh} there.

As before, specifying the arguments for \texttt{foobar} will be ignored if \texttt{foobar} isn't in the \texttt{machines} list.

Comments are \texttt{//}.

The list of available machines is aqua1, aqua2, aqua3, evan, felix, gus, hal, ida, jing, kirk, master, and manual. To add more, edit \texttt{Configuration.cpp}

\section{Modifying the Fleet Manager}
\subsection{Adding Views}
There are four places in \texttt{ui.cpp} that need to be modified to add a view; the help text, the table formatting, the navigation character handlers, and the view render block.
\begin{description}
\item[Help Text] Find the block in \texttt{UI::setTableFormats()} of additions to \texttt{m\_help["nav"]}; the syntax is a struct of three strings; 
\cmdline{\{view name, navigation character, help text description\}}
\item[Table Formats] Find the blocks in \texttt{UI::setTableFormats()} like

\cmdline{foo.push\_back("BLAH")}
\cmdline{foo.push\_back("BLAH BLAH")}
\cmdline{m\_headers["foobar"].push\_back(foo)}

The map \texttt{m\_headers} stores the headers for each view. A header is a vector of vectors of strings; the outer vector stores rows to feed to ACTables, and the inner vector stores the strings to put in each column of the table. Sections that have multiple header rows should be specified as

\cmdline{foo1.push\_back("BLAH"); foo2.push\_back("DUH")}
\cmdline{foo1.push\_back("BLAH BLAH"); foo2.push\_back("DUH DUH")}
\cmdline{m\_headers["foobar"].push\_back(foo1);}
\cmdline{m\_headers["foobar"].push\_back(foo2);}

the first block would result in a table formatted like

\begin{tabular}{l|l}
BLAH & BLAH BLAH\\
\hline
...&...\\
\end{tabular}

while the second block would result in a table formatted like

\begin{tabular}{l|l}
BLAH & BLAH BLAH\\
DUH & DUH DUH\\
\hline
...&...\\
\end{tabular}

Usually one or two lines is sufficient; the first line to delineate sections and the second line for column headers. Add your own section, consistent with what you put in the Help Text section. You will revisit this in Adding Topics to a View.

\item[Character Handlers] Find the block in \texttt{UI::actOnKeyPress()} with sequences like
\cmdline{else if (m\_key\_feed=="M") \{}
\cmdline{	m\_view = "MOOS";}
\cmdline{	command\_match = true;}
\cmdline{\}}
and add you own, consistent with the information you put in the Help Text section
\item[View Render] Find the block in \texttt{UI::printWindow()} that looks like 
\cmdline{if (m\_view=="FOO") \{}
\cmdline{	view\_table << something}
\cmdline{\}}
\cmdline{else if (m\_view=="BAR") \{}
\cmdline{	view\_table << something else}
\cmdline{\}}
\cmdline{...}
and add a similar block checking for your new view. See the next section for how to fill out that block.
\end{description}
\subsection{Adding Topics to a View}
There are two places in \texttt{ui.cpp} that need to be modified to add a column to a view; the table formatting block and the view rendering block.
\begin{description}
\item[Table Formats] Find your block in \texttt{UI::setTableFormats()}, the same as your Table Formats block from Adding Views. Add a string to all the inner vectors. Disallowed\footnote{Used by ACTables for formatting.} strings include "{\textbackslash}n" and "$|$", and allowable strings include "", "\textbackslash", "/", and "\#".
\item[View Render] Find your block in \texttt{UI::printWindow()}, the same as your View Render block from Adding Views. The \texttt{n}th line such as \texttt{view\_table << something} will fill the \texttt{n}th column of the table as ordered in Table Formats.
\end{description}
\subsection{Adding Commands}
The interface for the UI to call vehicle commands, to get information or to take action, is public \texttt{ManagedMoosMachine} methods.

Commands are fired off into the void, with a file to write results back to. These files are opened and read synchronously with the local machine, with a small proability\footnote{Determined by the duty cycle of file IO} of reading partially written data\footnote{uFleetManager's networking layer is written such that in that case, the message ID is the last thing written, and only once it is complete will the app do anything with that data. In formal terms, this satisfies only the Consistency pillar of CAP. If the user clears the cache agggressively, it also weakly satisfies Partition Tolerance.}. This architecture approximates threading\footnote{This architecture was selected in keeping with Dr. Benjamin's standing instructions that any user with basic C++ and Bash experience should be able to understand any code in the lab. The PAVLAB considers threading a non-basic feature.}, but does not require maintainers to understand threading per se.
\subsubsection{Dispatching}
At the high level, the fleet manager is a big wrapper for sending commands over \texttt{ssh};

\cmdline{ssh ADDR "remote\_cmd"}

The complexity in sending commands is in letting go of it so the app can return to its thread of execution. The normal way to execute commands from \texttt{C++} is with the \texttt{system\_call()} function\footnote{system\_call() has known security issues, be very careful using it unless you're absolutely sure you know the pedigree of the scripts you're calling with it.}, which is fine for local, synchronous commands. However, that naive approach is not sufficient to have many asynchronous commands in flight at the same time. The solution involves \texttt{nohup} (no hangup) and \texttt{\&}, and is implemented in the \texttt{\_dispatch()} function in \texttt{system\_call.cpp}\footnote{Note, I left some vestigal code about timeouts - I was leaking background processes, and attempting to solve that by sending out the scripts with a kill switch on a timer. Instead, the eventual solution was to use message indices and only send out a new request once the old one returned, I just haven't had time to clean up that bit of code.}.

The robust interface from \texttt{system\_call.cpp} is two functions, \texttt{system\_call\_dispatch\_pipe()} and \texttt{system\_call\_dispatch\_return()}. Both of them dispatch commands and capture an output from the script via \texttt{ssh} and write it to a named mailbox; \texttt{\_pipe()} captures from \texttt{stdout}, where \texttt{\_return()} captures the script's return value.

When you're adding a new public method to \texttt{ManagedMoosMachine}, it will be essentially a wrapper around either \texttt{system\_call\_dispatch\_pipe()} or \texttt{system\_call\_dispatch\_return()}. There are two common ways to do so; standard PAVLAB commands, and one-off commands. All else being equal, standard commands are preferable.

\subsubsection{Standard PAVLAB Commands}

The standard PAVLAB way of interfacing with lab machines outside of MOOS-IvP itself is via a \textit{machine}-common directory. Currently we are maintaining two lab \texttt{svn} repos, \texttt{pablo-common} and \texttt{mokai-common}, and they each contain a directory called FleetManagerScripts with the lab's standard scripts. Commands have a simple naming scheme: "pav\_\textit{action}\_\textit{object}.sh". Some examples are 

\begin{description}
\item \texttt{pav\_test\_ssh.sh}
\item \texttt{pav\_get\_svn\_rev\_moos.sh}
\item \texttt{pav\_up\_svn\_moos.sh}
\item \texttt{pav\_reboot\_computer.sh}
\end{description}

The sole exception is \texttt{pav\_not\_implemented.sh}, the placeholder implementation, which lacks a verb (such as "is").

All machines should have their relevant kind of \textit{machine}-common tree, but they will have the Anonymous Read Only version, \textit{machine}-common-aro\footnote{This allows anyone to call update and to use the scripts, but not to push changes. To push changes, talk to Dr. Benjamin about getting access to the \textit{machine}-common repos}.

There is a special helper function in \texttt{ManagedMoosMachine} to streamline calling those standard PAVLAB commands, \texttt{\_dispatchPavCmd()}. There are several good examples of its usage in the \texttt{ManagedMoosMachine} class.

\subsubsection{Special Commands}

Some commands do not lend themselves to the common and standardized system. For example, \texttt{ping} makes no sense to be hosted remotely. Rebooting computers can be configured to run without a password on some operating systems (e.g. Raspian) but it's not clear on others (e.g. Ubuntu) so one-off versions are sometimes needed.

In that case, compose the script in code and use the appropriate dispatcher (\texttt{system\_call\_dispatch\_pipe()} or \texttt{system\_call\_dispatch\_return()}) to run it. 

One interesting caveat is that unlike normal scripts, where instructions are separated by semicolons, in these scripts the instructions must be separated by newlines.

\subsubsection{Reading Mail}

Dispatched commands will, once they conclude, yield a result to a mailbox file. Mailbox files are files in the directory \texttt{/tmp/MOOSMAIL}. The usual naming is \texttt{/tmp/MOOSMAIL/}\textit{Machine}\texttt{\_}\textit{commandName}\texttt{.mailbox}. For ease of use, use the \texttt{ManagedMoosMachine} helper function \texttt{serviceMailboxName()}.

Once that result is put in the mailbox, you'll want to read it. At its core, we're just reading lines out of the mailbox and parsing them. By checking for message indices, this step also serves as a caching and synchronizing step.

Consider a new \texttt{ManagedMoosMachine} public method, \texttt{checkFooServiceMail()}; its implementation might look something like this:

\cmdline{vector<string> mail\_list = readServiceMailbox(m\_mail["fooService"].mailbox);}
\cmdline{index\_t index = grabIndex(mail\_list);}
\cmdline{if (receiveUpdate(m\_mail["fooService"].cache, index)) \{}
\cmdline{	// do parsing here...}
\cmdline{	m\_mail["fooService"].cache.data = /* a result string */}
\cmdline{\}}
\cmdline{return(get\_data\_and\_staleness(m\_mail["fooService"].cache));}

When returning a status, consider looking in the \texttt{Status} namespace in \texttt{Constants.h}. These statuses are shared throughout the app, allowing them to be reasoned over in the UI. 

\subsubsection{Required Variables and Caching}
The infrastucture of the \texttt{ManagedMoosMachine} is in the \texttt{m\_mail} map; it contains the cache (of type \texttt{StampedData}, refer to \texttt{Utils.h}) and the mailbox file name.

To add a new required variable, find the \texttt{ManagedMoosMachine} constructor, all you have to do is add a line like

\cmdline{m\_mail["foobar"].mailbox = serviceMailboxName("fooBar");}

near the others that follow that pattern.

Bracket lookup adds foobar to the \texttt{m\_mail} map, and then sets the name of the mailbox it uses for script writebacks. Below, it loops through all \texttt{m\_mail} entries (which now includes foobar) and adds a blank \texttt{StampedData} struct. After rebuilding and running, if you look in the mailbox directory (by default, \texttt{/tmp/MOOSMAIL}), you should now see an \texttt{Alpha\_fooBar.mailbox}, a \texttt{Bravo\_fooBar.mailbox}, etc. Their contents will be whatever you asked to be written back, plus an index at the end:
\cmdline{MOOS\_MANAGER\_MESSAGE\_INDEX:N}
which is part of the networking solution described in Adding Commands and Dispatching. Once you call \texttt{grabIndex()}, that line will be stripped out and you will only be presented with the payload.
\subsection{Miscellaneous}
\begin{enumerate}
\item To add new vehicles, look to \texttt{Configuration.cpp}
\end{enumerate}
\subsubsection{Tips and Tricks}
\begin{enumerate}
\item If you're getting strange status results, always check the actual contents of the mailbox:

\cmdline{cat /tmp/MOOSMAIL/that.mailbox}

or if you want to get really fancy

\cmdline{while true; do cat /tmp/MOOSMAIL/that.mailbox; sleep 1; clear; done}
\item If you're trying to debug script output that's going to stdout once and then disappearing, try adding a \texttt{sleep(seconds);} to the main render loop in \texttt{ui.cpp}. In fact, there's one already there commented out for just that eventuality. 
\end{enumerate}
\section{Configuring Machines to work with the Fleet Manager}
\subsection{SSH Keys}
In order to script commands over ssh, the target machine needs your ssh public key to allow you to log on with no password. 

The \textit{machine}\texttt{-common} repos have scripts that should update the target machine's \texttt{$\sim$/.ssh/authorized\_keys} with your public key if it's added into the \textit{machine}\texttt{-common/}\textit{machine}\texttt{\_authorized\_keys} file.

For machines that don't have a \textit{machine}\texttt{-common} repo, or if you can't get a hold of someone with editing privileges, you can add it in manually.
\subsection{Shell Startup and Sources}
In order to have the PAVLAB commands available both for interactive shells that you start when you log in and the non-interactive shells that \texttt{uFleetManager} starts, you need to source it in the right place in your \texttt{$\sim$/.profile} or \texttt{$\sim$/.bashrc} files. Many of them have blocks like:

\cmdline{\# If not running interactively, don't do anything}
\cmdline{case \$- in}
\cmdline{    *i*) ;;}
\cmdline{      *) return;;}
\cmdline{esac}

Your block must be before anything like that. Your block should look much like this:

\cmdline{\# Source the svn-controlled bashrc or cshrc}
\cmdline{if [ \$SHELL = "/bin/bash" ]; then}
\cmdline{    . "\$HOME/pablo-common-aro/dot\_bashrc"}
\cmdline{elif [ \$SHELL = "/bin/csh" ]; then}
\cmdline{    . "\$HOME/pablo-common-aro/dot\_cshrc"}
\cmdline{fi}

Substituting \texttt{pablo-common}, \texttt{mokai-common}, or \texttt{mokai-common-aro} as appropriate.

\subsection{Software}
All target machines should have a copy, ARO or otherwise, of at least one \textit{machine}\texttt{-common} tree like \texttt{pablo-common} or \texttt{mokai-common}.

Target machines should also have MOOS-IvP and any other MOOS-IvP trees (e.g. Aquaticus and Colregs) necessary for the missions.
\subsection{Permissions}
A few select commands require special permissions, namely shutting down and rebooting computers. Each operating system has different amounts that you can deprotect those two commands, and we have explored those for Raspbian and Ubuntu:
\begin{description}
\item[Raspbian] Run 

\cmdline{sudo visudo -f /etc/sudoers.d/shutdown}

and add the lines

\cmdline{student2680 host = (ALL) NOPASSWD: /sbin/shutdown now}
\cmdline{student2680 host = (ALL) NOPASSWD: /sbin/reboot}

then run

\cmdline{sudoedit /etc/polkit-1/localauthority/50-local.d/reboot.pkla}

and add the lines

\cmdline{[Shutdown]}
\cmdline{Identity=unix-user:student2680}
\cmdline{Action=org.freedesktop.login1.power-off}
\cmdline{ResultAny=yes}

\cmdline{[Reboot]}
\cmdline{Identity=unix-user:student2680}
\cmdline{Action=org.freedesktop.login1.reboot}
\cmdline{ResultAny=yes}

and now the two commands \texttt{shutdown now} and \texttt{reboot} are completely unprotected by \texttt{sudo}
\item[Ubuntu] Run

\cmdline{sudo visudo -f /etc/sudoers.d/shutdown}

and then add the lines

\cmdline{student ALL=NOPASSWD: /sbin/shutdown now}
\cmdline{student ALL=NOPASSWD: /sbin/reboot}

and now the two commands \texttt{shutdown now} and \texttt{reboot} can be run with \texttt{sudo}, so \texttt{sudo shutdown now} and \texttt{sudo reboot}, but will not require a password. When run over \texttt{ssh}, it still requires \texttt{sudo} in the command and the \texttt{-t} argument to \texttt{ssh}, but will not require a password.
\end{description}
\section{Debugging}
\begin{tabular}{l|l|l}
Symptom & Likely Issue & Resolution\\
\hline
Semicolons in command (see History) & Needs to be {\textbackslash}n & Replace in relevant\\&& ManagedMoosMachine dispatcher function.\\
App responds slowly & Resource starved & Check CPU usage\\
&& Check process count\\
&& Try closing Google Chrome\\
&& or other resource-hogging programs\\
&& Restart uFleetManager\\
No machines seen & On wrong network & Switch to kayak-local\\
& Wifi is down & Check that the Bullet is on and working\\
Password requests in \texttt{stdout} & Target needs ssh keys & Add your public key to \\ 
&&\texttt{$\sim$/.ssh/authorized\_keys} on the\\
&& target machine
\end{tabular}
\end{document}